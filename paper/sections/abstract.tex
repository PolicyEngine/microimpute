\section*{Abstract}

To address the impossibility of peforming policy analysis that relies on wealth wihout a comprehnsive datataset capturing representative demographic, income and asset data across records, this paper evaluates the methodological advantages of Quantile Regression Forests (QRF) for wealth imputation from the Survey of Consumer Finances (SCF) to the Current Population Survey (CPS). We demonstrate that QRF outperforms traditional imputation approaches by preserving conditional distributions rather than merely conditional means, a critical distinction that makes distributional analyses of policy reforms under microdata more accurate. Our empirical analysis, implemented through the $\texttt{microimpute}$ package, provides evidence that QRF reduces bias in wealth imputations, achieving a 20.5\% reduction in average quantile loss (a measure of error against the full conditional distribution instead of central estimates) compared to OLS, 14.8\% compared to Hot Deck Matching and a 6\% reduction compared to Quantile Regression. Using 5-fold cross-validation on 22,975 SCF households, QRF achieves an average quantile loss across all quantiles of 6.6m, demonstrating superior distributional accuracy, particularly in the 10th-80th percentile range. While we focus on wealth in our experimental approach, these results suggest QRF's advantages extend to any skewed variable requiring distributional preservation, including consumption, medical expenses, and other heavy-tailed economic measures. These technical improvements have substantial implications for microdata enhancement and fiscal policy research, enabling analysis from wealth tax values and asset-dependent SNAP or Medicaid qualifications to policy impact across wealth deciles. We release our open-source $\texttt{microimpute}$ package to facilitate microimputation across the field, providing automated method comparison, hyperparameter tuning, and survey weight integration capabilities that streamline the imputation workflow for complex survey data.