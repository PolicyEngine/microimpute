\section{Conclusion}

This paper has reviewed methodological advances in microdata imputation, focusing on techniques suitable for complex survey data like wealth. We highlighted the limitations of traditional methods and the advantages of novel approaches, particularly Quantile Regression Forests, for handling skewed distributions and non-linearities.

The key contributions of this work are the synthesis of current knowledge on imputation for challenging microdata, a detailed examination of QRF's suitability, and an introduction to the $\texttt{microimpute}$ package. Our review suggests that QRF represents a step forward in preserving the statistical integrity of imputed microdata, increasing the robustness of economic analysis. The implementation of QRF in the $\texttt{microimpute}$ package provides a practical tool for researchers seeking to combine detailed microdata across datasets.

Future research should continue to refine QRF for imputation, particularly in response to challenges like limited data at extreme quantiles. Comparative studies against other emerging techniques, like deep learning models \citep{alaa2024deep}, are also vital. Continued innovation in imputation methodology supports the reliability of evidence-based research and policymaking.