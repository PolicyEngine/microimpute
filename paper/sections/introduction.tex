\section{Introduction}

Microsimulation models and detailed microdata analyses are essential tools for understanding the distributional impacts of policies and social changes. These analyses require data that accurately represent both the demographic composition of a population and its economic circumstances. However, available data sources, particularly large-scale surveys, often suffer from missing data due to item nonresponse \citep{dempster1983introduction} or as a result of survey design. If not appropriately addressed, missing data can introduce substantial bias, undermining the validity of research conclusions \citep{graham2009missing} or limiting analysis opportunities altogether, increasing the risk of ineffective policy decisions and unintended consequences.

This problem extends beyond item nonresponse to systematic underreporting of certain income types in surveys. Evidence from fiscal and financial surveys in the UK demonstrate how data imputation can improve data quality. Dividend income in the Family Resources Survey is severely underreported, with the survey not even collecting data about directors' dividend incomes until the 2021-2022 survey year \citep{dwp2023frs}. The UK's "SPI adjustment" methodology addresses this by replacing survey responses with administrative tax data for high earners, revealing income distributions previously hidden by measurement error \citep{advani2023measuring}. These examples illustrate that imputation from administrative sources does not merely fill gaps but can provide superior data quality compared to self-reported survey responses, particularly for sensitive financial variables prone to underreporting.

Traditional imputation approaches struggle with wealth data's right-skewness, heavy tails, and non-linear relationships with demographic and economic predictors. Wealth's heterogeneous interaction with income complicates imputation. Advani and Summers \citep{advani2020capital} demonstrated that capital gains, a key component of wealth changes, are distributed across the income spectrum with substantial volatility, finding that even individuals at the 80th income percentile have only a 1\% probability of realizing any taxable gains, while those who do receive gains show extreme variability in amounts. These characteristics fundamentally violate assumptions underpinning conventional methods like Ordinary Least Squares (OLS) and Quantile Regression, resulting in significant distortions that undermine policy analysis \citep{meinshausen2006quantile}.

This paper demonstrates that Quantile Regression Forests (QRF) provides superior performance for wealth imputation between the Survey of Consumer Finances (SCF) and the Current Population Survey (CPS). By modelling entire conditional distributions rather than conditional means alone, QRF preserves critical distributional features of wealth data. We implement this approach through the $\texttt{microimpute}$ package, a specialised tool developed for survey data imputation that provides a complete pipeline for imputation and analysis, tailored to the dataset at hand. This package automates the comparison of four imputation methods, namely QRF, OLS, Hot Deck Matching, and Quantile Regression, automatically selecting the one achieving lowest average quantile loss to perform the final wealth impuation.

The remainder of this paper is organized as follows: Section 2 reviews the statistical properties of wealth microdata and the evolution of imputation techniques in the literature, discussing the strengths and limitations of the four methods evaluated. Section 3 describes our data sources (SCF and CPS) and their characteristics. Section 4 presents the $\texttt{microimpute}$ package in detail. Section 5 presents our empirical results. Section 6 discusses implications and limitations, and Section 7 concludes.
Our analysis makes two key contributions:
\begin{itemize}
    \item An open-source microimputation package that facilitates the evaluation of multiple imputation methods tailored to specific dataset needs
    \item A validation framework comparing novel methodological approaches to traditional imputation methods demonstrated on statistically challenging data like wealth distributions
    \item A demonstration of QRF's advantages for wealth imputation, achieving better distributional estimates and a reduction in average quantile loss compared to traditional methods 
\end{itemize}