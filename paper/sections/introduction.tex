\section{Introduction}

Microsimulation models and detailed microdata analyses are essential tools for understanding the distributional impacts of policies and social changes. These analyses require data that accurately represent both the demographic composition of a population and its economic circumstances. However, available data sources, particularly large-scale surveys, often suffer from missing data due to unit or item nonresponse \citep{dempster1983introduction}. If not appropriately addressed, missing data can introduce substantial bias and reduce statistical power, undermining the validity of research conclusions \citep{graham2009missing}. This problem is especially acute for complex variables like household wealth, which are often characterized by high skewness, influential outliers, and significant nonresponse for sensitive items \citep{barcelo2008impact}. UK evidence confirms these challenges: the Office for National Statistics acknowledges that household surveys struggle to capture the incomes of the very richest individuals, particularly those in the top 1\% \citep{ons2019using}, while the Institute for Fiscal Studies estimates that nearly £800 billion of wealth held by the wealthiest UK households is missing from survey data \citep{crawford2016distribution}. Furthermore, declining survey response rates—falling from 49\% pre-pandemic to 31\% in recent years—compound these measurement challenges \citep{ifs2025new}.

Traditional imputation approaches struggle with wealth data's extreme right-skewness, heavy tails, and complex non-linear relationships with demographic and economic predictors. These characteristics fundamentally violate assumptions underpinning conventional methods like Ordinary Least Squares (OLS) and Quantile Regression, resulting in significant distortions that undermine policy analysis \citep{meinshausen2006quantile}.

This paper demonstrates that Quantile Regression Forests (QRF) provides superior performance for wealth imputation between the Survey of Consumer Finances (SCF) and the Current Population Survey (CPS). By modelling entire conditional distributions rather than conditional means alone, QRF preserves critical distributional features of wealth data. We implement this approach through the Microimpute package, a specialised tool developed for survey data imputation that provides a complete pipeline for imputation and analysis, tailored to the dataset at hand.

The remainder of this paper is organized as follows: Section 2 reviews the statistical properties of wealth microdata and the evolution of imputation techniques in the literature. Section 3 describes our data sources (SCF and CPS) and their characteristics. Section 4 presents the QRF methodology and the Microimpute package in detail. Section 5 presents our empirical results. Section 6 discusses implications and limitations, and Section 7 concludes.
Our analysis makes two key contributions:
\begin{itemize}
    \item An open-source microimputation package that facilitates the evaluation of multiple imputation methods tailored to specific dataset needs
    \item A validation framework comparing novel methodological approaches to traditional imputation methods demonstrated on statistically challenging data like wealth distributions
\end{itemize}